\documentclass{article}
\usepackage{graphicx} % Required for inserting images

%Here's some stuff that may help later

\usepackage{mathrsfs}
\usepackage{amscd}
\usepackage{amsmath}
\usepackage{amssymb}
\usepackage{amsthm}
\usepackage{epsf}
\usepackage{latexsym}
\usepackage{verbatim}
\usepackage[all, cmtip]{xy}
\usepackage{tikz}
\usetikzlibrary{positioning}
\usetikzlibrary{matrix}
\usepackage{float}
%\usepackage{showkeys}
\usepackage{hyperref}
\usepackage{comment}
\usepackage{enumitem}
\usepackage{mathtools}


%Dont forget these Tableaux macros
\tikzstyle{bsq}=[rectangle, draw, thick, minimum width=.5cm, minimum height=.5cm]
\tikzstyle{bver}=[rectangle, draw, thick, minimum width=1cm, minimum height=2cm]
\tikzstyle{bhor}=[rectangle, draw, thick, minimum width=2cm, minimum height=1cm]

\usepackage[left=3.2cm, right=3.2cm]{geometry}

\setlength{\textheight}{8.5in} \setlength{\topmargin}{0.0in}
%\setlength{\textwidth}{6in} \setlength{\leftmargin}{-.5in}
\setlength{\headheight}{0in} \setlength{\headsep}{0.3in}
\setlength{\leftmargin}{1.5in}

\newtheorem{theorem}{Theorem}[section]
\newtheorem{lemma}[theorem]{Lemma}
\newtheorem{conjecture}[theorem]{Conjecture}
\newtheorem{corollary}[theorem]{Corollary}
\newtheorem{proposition}[theorem]{Proposition}
\newtheorem{question}[theorem]{Question}
\newtheorem{varexample}[theorem]{Example}



\theoremstyle{definition}
\newtheorem{remark}[theorem]{Remark}
\newtheorem{definition}[theorem]{Definition}

%%%%%%%%%%%%%%%%%%%%%%%%%%%%%%%%%%%%%%%%%%%%%%%%%%%%

\title{MathLab Write Up}
\author{Sam Dalton, Cameron Egbert, Ayham Yousef}
\date{July 2024}

\begin{document}

\maketitle

\section{Introduction}

\begin{definition}
    Let G be a multi-spoked graph with center vertex $v_0$ and additional vertices $v_i$ along each corresponding spoke of weight $\ell_i$. All edges are of the form $(v_0,v_i)$. Let $c_i$ denote the number of chips placed on any vertex $v_i$.
\end{definition}

\begin{definition}
    Chip-firing is the operation in which chips placed on vertices are then transferred to adjacent vertices through the weighted edges. The weight of the edges 
\end{definition}



\begin{lemma}
    Let $G$ be a spoke graph on $n$ vertices with edges weighted $\ell_i$. Let $D$ be a winning divisor for $r$-gonality with the necessary assumptions. If $\ell_j>k$, then there are at least $r$ chips on $v_j$.
\end{lemma}
\begin{proof}
        Given k chips for $v_m$, and weight $\ell_i > k$, it is not possible to fire from $v_0$ to $v_1$ without creating new debt of $k-\ell_i$ on $v_0$. Since chip firing from $v_0$ to $v_i$ creates debt $\Rightarrow$ $\nexists$ any chip firing operation between $v_0,v_i |c_0,c_i\geq 0$. Since $\nexists$ any valid chip-firing operations, there must be at least r chips on $v_i$ in order to account for the case where r chips are removed from $v_i$. 

\end{proof}


\begin{lemma}
    Let $G$ be a spoke graph on $n$ vertices with edges weighted $\ell_i$. Let $A_k = \{\ell_i \mid \ell_i>k \}$.  
    $$gon(G) = \min_{k}\{k + |A_k|\}$$
\end{lemma}
\begin{proof}
    
\end{proof} 

\begin{lemma}
    Let $G$ be a spoke graph on $n$ vertices with edges weighted $\ell_i$. Let $A_{1,k} = \{\ell_i \mid \frac{k}{2}<\ell_i\leq k \}$ and $A_{2,k} = \{ \ell_i \mid \ell_i>k \}$.  
    $$gon_2(G) =\begin{cases}
         \min_{k}\{k + |A_{1,k}| +2|A_{2,k}|\} & \text{ if } \max\{\ell_i \mid \ell_i \leq \frac{k}{2}\} +\min\{ \ell_i \mid \frac{k}{2}<\ell_i\leq k \}>k \\
         \min_{k}\{k + |A_{1,k}|-1 +2|A_{2,k}|\} & \text{ if } \max\{\ell_i \mid \ell_i \leq \frac{k}{2}\} +\min\{ \ell_i \mid \frac{k}{2}<\ell_i\leq k \}\leq k \\
    \end{cases}$$
\end{lemma}
\begin{proof}
    
\end{proof}

\begin{lemma}
    Let $G$ be a spoke graph on $n$ vertices with edges weighted $\ell_i$. Let $D$ be a winning divisor for $r$-gonality with the necessary assumptions. If $\ell_j=k-a$ for some $a\in \mathbb{Z}_{\geq 0}$, then there are at least $r-(a+1)$ chips on $v_j$.
\end{lemma}
\begin{proof}
    
\end{proof}

\begin{lemma}
    Let $G$ be the spoke graph with $n-1$ spokes and $\ell_i = i$ for all $1\leq i \leq n-1$. Then
    $$gon(G) = $$
\end{lemma}

\begin{lemma}
    Let $G$ be the spoke graph with $n-1$ spokes and $\ell_i = i$ for all $1\leq i \leq n-1$. Then
    $$gon_2(G) = \left\lfloor \frac{3}{2} gon(G) \right\rfloor$$
\end{lemma}

\begin{lemma}
    Let $D$ be a winning divisor
\end{lemma}

\end{document}
