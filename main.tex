\documentclass[12pt]{article}
 
\usepackage[margin=1in]{geometry} 
\usepackage{amsmath,amsthm,amssymb, graphicx, multicol, array, tikz}
 
\newcommand{\N}{\mathbb{N}}
\newcommand{\Z}{\mathbb{Z}}
\newcommand{\R}{\mathbb{R}}
\newcommand{\Q}{\mathbb{Q}}
\newcommand{\C}{\mathbb{C}}

\newtheorem{theorem}{Theorem}
 
\newenvironment{problem}[2][Problem]{\begin{trivlist}
\item[\hskip \labelsep {\bfseries #1}\hskip \labelsep {\bfseries #2.}]}{\end{trivlist}}

\begin{document}
 
\title{MathLab \#8}
\author{Cameron Egbert\\
Summer 2024}
\date{Tue July 1st}
\maketitle


\begin{problem}{1}
    For a multicycle graph $G = \{V, E\}$, $gon(G) = 2$ if and only if $e_{12}=e_{34}=1$ or $e_{23}=e_{41}=1$.
\end{problem}
\begin{proof}
    If $e_{12}=e_{34}=1$, the gonality of the graph is two because the starting values, for each respective vertex, $\{1, 0, 0, 1\}$, satisfy gonality 2 and if $e_{23}=e_{41}=1$, the gonality of the graph is two because the starting values $\{1, 1, 0, 0\}$, for each respective vertex,  satisfy gonality 2. Additionally, if $e_{12}=e_{34}=1$ and $e_{23}=e_{41}=1$ are not true, then there must exist a node $v_j$ such that $v_{ij} > 1$ and $v_{jk} > 1$ and a graph, of size 4, containing a node like this must be gonality greater than 2. This is because $v_j$ must have a starting value greater than zero as any a chip firing sequence would require at least 2 chips on $v_i$ and on $v_j$ to rectify a negative $v_j$. Additionally with $v_j$ starting with 1 chip, there is no way to add less than two chips to satisfy gonality, no matter which node you put the singlular chip on, (there are counterexaples for each). Therefore proved.
\end{proof}

\begin{problem}{2}
    There are no winning configurations smaller than 
    \[
        \min_{k} (l_k + |l_i > l_k)|)
    \] 
    in a multi-spoke graph.
\end{problem}
\begin{proof}


    
    Given the gonality $<$ n, there will exist nodes in a winning divisor with starting value 0. We can build minimum divisors for all configurations of 0-nodes and show they are values found in instances of k for the term in $\min_{k}$ (whatever).

    If center is 0, 
    If center isn't 0: need largest spoke 0 to transfer back to middle
    
\end{proof}

\end{document}